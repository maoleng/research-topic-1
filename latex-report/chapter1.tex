\section{Background and Motivation}
\hspace{\parindent}
In recent years, the proliferation of online transactions has led to an exponential increase in the volume of digital payments worldwide. While this surge in digital commerce has revolutionized the way we conduct business, it has also opened new avenues for fraudulent activities. According to recent studies, online payment fraud is estimated to cost businesses billions of dollars annually, posing significant financial losses and reputational damage.\\

Traditional fraud detection methods, primarily based on rule-based systems and static thresholds, have become inadequate in dealing with the sophisticated tactics employed by fraudsters. These methods often struggle to adapt to evolving fraud patterns and struggle to distinguish between legitimate and fraudulent transactions accurately.\\

To address these challenges, there has been a growing interest in leveraging advanced machine learning techniques, particularly Graph Neural Networks (GNNs), for fraud detection in online payments. GNNs offer a promising approach by modeling the complex relationships and interactions inherent in payment networks, thus capturing the intricate patterns indicative of fraudulent behavior.\\

The motivation behind this project stems from the need to develop more robust and adaptive fraud detection systems capable of mitigating the risks associated with online payments. By harnessing the power of GNNs, we aim to enhance the accuracy and efficiency of fraud detection while minimizing false positives and false negatives. Ultimately, our goal is to contribute to the development of scalable, data-driven solutions that can effectively combat fraudulent activities in online transactions.\\

In this report, we present our approach to detecting fraud in online payments using Graph Neural Networks. We delve into the methodology employed, the dataset utilized, the experimental results obtained, and the implications of our findings. Through this research endeavor, we seek to provide insights into the potential of GNNs in bolstering the security and integrity of digital payment ecosystems.\\

\section{Objectives of the Project}
\hspace{\parindent}
Developing a Fraud Detection System: The foremost objective is to design and implement a robust fraud detection system tailored specifically for online payment networks. This system will leverage Graph Neural Networks (GNNs) to effectively capture the complex relationships and patterns indicative of fraudulent behavior within the payment ecosystem.\\

Exploring Graph Representation: We aim to explore various methods of representing online payment networks as graphs, including node and edge attributes, graph construction techniques, and feature engineering. By representing the payment data in a graph structure, we seek to exploit the inherent relational information to enhance the accuracy of fraud detection.\\

Utilizing Graph Neural Networks: The project aims to investigate the effectiveness of Graph Neural Networks in detecting fraudulent transactions within online payment networks. We will explore different GNN architectures, including graph convolutional networks (GCNs) and graph attention networks (GATs), to identify the most suitable model for our specific use case.\\

Training and Evaluation: Our objective is to develop robust training strategies for the GNN-based fraud detection system, including data preprocessing, model training, and validation techniques. We will also establish comprehensive evaluation metrics to assess the performance of the system accurately.\\

Performance Benchmarking: Another objective is to benchmark the performance of the proposed GNN-based fraud detection system against traditional machine learning approaches and baseline models. By conducting comparative analyses, we aim to demonstrate the superiority of GNNs in detecting fraudulent activities in online payments.\\

